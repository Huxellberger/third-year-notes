% Set the author and title of the compiled pdf
\hypersetup{
  pdftitle = {\Title},
  pdfauthor = {\Author}
}

\section{Finite precision computation}

Unfortunately, the world is not solely restricted to integers, and computers
often need to work with real numbers $\mathbb{R}$. With integers, the main
problem we have in computer terms is overflow, and since there is a finite
distance from one to the next, they are easy to encode in a computer.

On the other hand, between any two real numbers, there are infinitely many more
real numbers. Since computers are discrete, we need to sample the real numbers
so that we can find a representation for them in the computer. However, this
introduces errors, since we can't represent every value exactly and therefore
most approximate.

\subsection{Floating point numbers}

%Slide 1

One problem we have with computation is that we don't know what the error is
with computations; how `good' is the result of an algorithm or computation? We
would like to know the error bounds of a solution, and have the output be
reliable.

%Slide 2

In the $70$'s, it was realised that different floating point implementations
produced different results. This had significant concerns for reproducability,
and as a result the ANSII IEEE standard for binary floating point arithmetic was
created.

%Slide 3

Each floating point number is represented as four integers; the base, the
precision, the exponent and the mantissa.

\[
  x = \pm m \times b^{e-n}
\]

Where:

\begin{center}
  \begin{tabular}{>{$}l<{$}|l}
    m & The Mantissa (the bit before the decimal place)\\
    b & The base (or radix), usually two or ten\\
    e & The exponent (the power of the radix)\\
    n & The precision (the numbewr of digits in the mantissa)
  \end{tabular}
\end{center}

We can represent different numbers in different ways, for example:

\[
  0.121e10^3 = 0.0121e10^4
\]

In this case, we can normalise the way in which we represent numbers and at
least all computers will get the same errors.

% Slide 4

The amount of numbers we can represent with the floating points depends on the
values permissable for $b,n$ and $e$. When $b=2,n=2$ and $e=[-2\dots2]$:

\marginpar{The Python code used to generate this is found in the
\texttt{/COMP36212/programs} folder of the source for these notes.}

\begin{center}
  \begin{tabular}{>{$}c<{$} >{$}c<{$} >{$}c<{$} >{$}c<{$} >{$}c<{$}}
    2 & \times & 2^{-2} & = & 0.500000\\
    3 & \times & 2^{-2} & = & 0.750000\\
    2 & \times & 2^{-1} & = & 1.000000\\
    3 & \times & 2^{-1} & = & 1.500000\\
    2 & \times & 2^{0} & = & 2.000000\\
    3 & \times & 2^{0} & = & 3.000000\\
    2 & \times & 2^{1} & = & 4.000000\\
    3 & \times & 2^{1} & = & 6.000000\\
    2 & \times & 2^{2} & = & 8.000000\\
    3 & \times & 2^{2} & = & 12.000000\\
  \end{tabular}
\end{center}

The mantissa is always $2$ or $3$ since we're using an explicit one, so the
binary values are either $[1,0]$ or $[1,1]$.

Floating point numbers are relatively spaced; even though they might not be the
same distance apart, the ratio between them is the same. The unit round off
(basically the last digit) is called the \textit{relative machine precision}.

The \textbf{Relative Machine Precision} is given by $u = 0.5 \times b^{1 - n}$,
and is the largest possible difference between a real number and its floating
point representation. In the above example, $u = 0.5 \times 2^{-1} =
\frac{1}{4}$. The value $2u$ is called the \textbf{Machine Precision}.

%Slide 5

In the exam, assume explicit storage of leading bit of mantissa.

\subsubsection{Real world floats}

% TODO: Insert diagram of float format here

For the sign bit, $0$ is for positive numbers and $1$ is for negative ones. The
exponent must also be able to represent negative numbers (in the case of
$24\times2^{-2}$ for example), and thus in single precision floats, a bias of
$+127$ is added to the exponent and that value is stored. The exponent values
$-127$ and $+128$ are reserved for special numbers.

The first bit of the mantissa is implicitly $1$ in the IEEE base two floating
point representation. This is because normalised numbers always have $1$ as the
first digit of their mantissa, and then we can get another digit of precision.

% TODO: Ranges of floating point numbers slide 1

Sixty-four bit floating point numbers have one sign bit, $11$ exponent bits and
$52$ mantissa bits. This means their bias will be $2^11 = 2048$ and the range
will be $2 - 2^{52} \times 2^{2^{11}}$.

The standard also has special values built in:

\begin{description}
  \item \textbf{Zero}: When the exponent is all zeros and the mantissa equal to 
  zero.
  \item \textbf{Denormalised number}: If the exponent is all zero, but the
  mantissa is non-zero, then the number is
  $-1^{sign} \times 0.m\times 2^{-126}$.
  \item \textbf{Infinity}: Exponent is all 1's and mantissa is all 0's. The
  sign dictates between positive and negative inifinity.
  %TODO: Signalling and quite NaN's
  \item \textbf{NaN}: Not your grandma, this is when the value isn't a real
  number, such as when a dividion by zero occurs.The exponent is all 1's and
  the mantissa is non-zero.
\end{description}

% TODO: Special operation flashcards

\subsubsection{Error in floating point numbers}
% Example exam q:
% Given a decimal number x, convert it to normal form and then into binary
% with specific values for e, m and b. Then trunctuate/round it and estimate
% the error

% TODO: How can a computer work out this error? It only knows floats...

When a real number is converted to floating point number, it may lose precision.
If the real number if $x$ and the floating point representation is
$\bar{x}$, then the error is:

\[
  e = \bar{x} - x
\]

You can find how many significant digits a floating point number approximates a
real number to by doing:

\[
  |\bar{x} - x| = 10^{-\text{significant digits}}
\]


However, the absolute error $e$ does not give us a very good description of the
accuracy (if the error is $10^{-6}$ but the value is $10^{-7}$ then we're very,
very inaccurate)! To rectify this, we have relative error:

\[
  r = \frac{e}{x} = \frac{\bar{x} - x}{x}
\]

%TODO: Wat is bounds?!

When we're getting the floating point number from the real number, we can
trucate the (possibly infinate) digts so that it fits in the mantissa. Simply
chopping the number so it fits in $m$ bits is called \textbf{simple trucation}.

If we round numbers instead of using simple truncation, then we can reduce the
error. We need some rules though:

\begin{itemize}
  \item If the part of the mantissa to be chopped of is less than $0.5$, use
  simple truncation.
  \item If it's greater, then increment the last digit of the mantissa, and then
  truncate.
  \item If it's equal to $0.5$, then we can do either (though IEEE says to round
  up).
\end{itemize}

Now the relative error is:

\[
  |r| = \frac{|e|}{|x|} = \frac{0.5 \times b^{e-n}}{|m|\times b}
      = \frac{1}{2 \times m \times b^n}
\]

There are three types of errors that computers can make:

\begin{itemize}
  \item Essential errors are ones that cannot be avoided (e.g. from erroneous
  input).
  \item Rounding errors are when we have to approximate real numbers with
  floating point ones. This error can be measured and controlled.
  \item Methodology errors come into play when replacing one problem by another
  similar, easier but less accurate problem is done. The solution is close, but
  not exact.
\end{itemize}

Unfortunately, errors can propogate through a computation. We must know the
errors introduced by every operation a computer performs on floating point
numbers. If we know $e_x = \bar{x} - x$ and $e_y = \bar{y} - y$, what is $e_{x
\cdot y}$?

The error introduced by addition, subtraction, multiplication and division is:

\begin{description}
  \item Addition:
  \[
    \bar{x} + \bar{y} = (x + e_x) + (y + e_y) = (e_x + e_y) + (x + y)
  \]
  \[
    e_{x+y} = e_x + e_y
  \]
  \item Subtraction:
  \[
    e_{x-y} = e_x - e_y
  \]
  \item Multiplication:
  \[
    \bar{x} \times \bar{y} = (x + e_x) \times (y + e_y) = xy + xe_y + ye_x + e_xe_y
  \]
  \[
    e_{x \times y} \equiv xe_y + ye_x
  \]
  \item Division:
  %TODO: Equation here...
  \[
    e_\frac{x}{y} \equiv \frac{1}{y}e_x - \frac{x}{y^2}e_y
  \]
\end{description}

%TODO: Relative error