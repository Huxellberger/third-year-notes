% Set the author and title of the compiled pdf
\hypersetup{
  pdftitle = {\Title},
  pdfauthor = {\Author}
}

\section{Finite precision computation}

Unfortunately, the world is not solely restricted to integers, and computers
often need to work with real numbers $\mathbb{R}$. With integers, the main
problem we have in computer terms is overflow, and since there is a finite
distance from one to the next, they are easy to encode in a computer.

On the other hand, between any two real numbers, there are infinitely many more
real numbers. Since computers are discrete, we need to sample the real numbers
so that we can find a representation for them in the computer. However, this
introduces errors, since we can't represent every value exactly and therefore
most approximate.

\subsection{Floating point numbers}

%Slide 1
One problem we have with computation is that we don't know what the error is with computations; how good is the result of an algorithm or computation?

%Slide 2
In the $70$'s, it was realised that different floating point implementations produced different results. This had significant concerns for reproducability, and as a result the ANSII IEEE standard for binary floating point arithmetic was created.

%Slide 3

Each floating point number is represented as four integers; the base, the precision, the exponent and the mantissa.

%TODO: Equation here

We can represent different numbers in different ways, for example:

\[
  0.121e10^3 = 0.0121e10^4
\]

In this case, we can normalise the way in which we represent numbers and at
least all computers will get the same errors.

% Slide 4

% TODO: Many examples of normalising numbers and putting them into binary
% This is homework answer on slide 4

Floating point numbers are relatively spaced; even though they might not be the
same distance apart, the ratio between them is the same. The unit round off
(basically the last digit) is called the \textit{relative machine precision}.

%Slide 5

In the exam, assume explicit storage of leading bit of mantissa

%Slide 6
