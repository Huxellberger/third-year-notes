\section*{Overview}

This course will explore certain classes of algorithms for modelling and
analysing complex systems, as arising in nature and engineering. These examples
include: flocking algorithms - e.g., how schools of fish or flocks of birds
synchronised; optimisation algorithms; stability and accuracy in numerical
algorithms.

\section*{Aims}

By the end of the course, students should:
\begin{itemize}
\item Appreciate the role of using nature-inspired algorithms in
computationally hard problems.
\item Be able to apply what they learnt across different disciplines.
\item Appreciate the emergence of complex behaviours in networks not present in
the individual network elements.
\end{itemize}

\section*{Syllabus}

\begin{mymulticols}
  \begin{itemize}
  \item PART 1: OPTIMISATION AND NATURE-INSPIRED ALGORITHMS (8 hours)
    \begin{itemize}
      \item Introduction to optimisation algorithms (2 hours)
	applications for optimisation algorithms
	local and global optimisation
	methods based on derivatives
	1.4 - direct search methods
      \item Stochastic optimisation (2 hours)
	grid search, random searches, multistart
	Simulated annealing
	particle swarm
      \item Evolutionary algorithms (4 hours)
	3.1 - basics of genetics and evolution
	3.2 - evolutionary programming
	3.3 - genetic algorithms
	3.4 - evolution strategies
	3.5 - genetic programming
        \end{itemize}
  \item PART 2: COMPLEX NETWORKS AND COLLECTIVE BEHAVIOUR (8 hours)
    \begin{itemize}
      \item Complex networks are groups of systems (normally, a big number of them) interconnected in a non-trivial and non-regular way
Introduction to complex networks (2 hours)
	Where are the networks and the complexity?
	Characterisation of complex networks
	Basic network properties and terminology. Topological analysis
      \item Complex network models. The structure of the network (2 hours)
	Regular networks
	Random-graph networks
	Small-world networks
	Scale-free networks
      \item Network dynamics and collective behaviour (3 hours)
	Distributed/local versus centralised/global
	The concept of self-organisation
	Synchronisation in complex dynamical networks
	Consensus over complex networks
	Swarm dynamics
	Consensus protocols
	Flocking algorithms
        \end{itemize}
  \item PART 3: NUMERICAL STABILITY AND ACCURACY OF COMPUTATIONS (8 hours)
    \begin{itemize}
      \item Introduction to finite precision computation (2 hours)
      \item Floating point arithmetic (examples that include error analysis insummation, evaluation of polynomials, recurrences, and basic linear algebra)(3 hours)
      \item Mixed precision algorithms (basic concept of iterative refinement,different speed of execution on different architectures, linear algebraexamples) (1 hour)
      \item Numerical solution of initial value problems (explicit/implicit methods,multistep methods, consistency, stability, convergence) (2 hours)
    \end{itemize}
  \end{itemize}

\end{mymulticols}

\section*{Attribution}

These notes are based off of both the course notes (found on
Blackboard). Thanks to the course staff (Eva Navarro-Lopez, Milan
Mihajlovic and Pedro Mendes) for such a good course! If you find any
errors, then I'd love to hear about them.

\section*{Contribution}

Pull requests are very welcome:
\url{https://github.com/Todd-Davies/third-year-notes}
