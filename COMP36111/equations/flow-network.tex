\documentclass{standalone}
\usepackage{tikz}

\usepackage{amsmath}
\usepackage{amssymb}

\usetikzlibrary{positioning}

\begin{document}

% flow network = \lbracket V, E, s, t, c\rbracket
% Where V, E is a directed graph with no back loops
%       s \in V
%       t \in V and s \notequal t
%       c : E \rightarrow \mathbb{N}
%       

\tikz[baseline]{
    \node (start) {flow network $ = ($};
    \node[draw=red,rounded corners,right = 0cm of start] (m1) {$V,E,$};
    \node[above = 0.8cm of m1] (l1) {A directed graph with no back-loops};
    \draw[-,red] (l1) -- (m1);
    %
    \node[draw=red,rounded corners,right = 0.2cm of m1] (m2) {$s,$};
    \node[below left of=m2] (l2) {The start node ($s \in V$)};
    \draw[-,red] (l2) -- (m2);
    %
    \node[draw=red,rounded corners,right = 0.2cm of m2] (m3) {$t,$};
    \node[above right of=m3] (l3) {The end node ($t \in V$, $t \neq s$)};
    \draw[-,red] (l3) -- (m3);
    %
    \node[draw=red,rounded corners,right = 0.2cm of m3] (m4) {$c$};
    \node[below = 0.8cm of m4] (l4) {The weight of each edge
      ($c : E \rightarrow \mathbb{N}$)};
    \draw[-,red] (l4) -- (m4);
    %
    \node[right = 0.1cm of m4] (end) {$)$};
}

\end{document}