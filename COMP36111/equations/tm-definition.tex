\documentclass{standalone}
\usepackage{tikz}

\usetikzlibrary{positioning}

\begin{document}

% TM = \langle K, \sigma, Q, q_0, T\rangle
% Where K >= 2 (num tapes)
%       sigma = non-empty finite set (tape alphabet)
%       Q = non-empty finite set (set of states)
%       q_0 = initial state
%       T = state transitions

\tikz[baseline]{
    \node (start) {$TM = \langle$};
    \node[draw=red,rounded corners,right of=start] (m1) {$K,$};
    \node[above left of=m1] (l1) {K $\geq 2$ (number of tapes)};
    \draw[-,red] (l1) -- (m1);
    %
    \node[draw=red,rounded corners,right = 0.2cm of m1] (m2) {$\Sigma,$};
    \node[below left of=m2] (l2) {non-empty finite set (tape alphabet)};
    \draw[-,red] (l2) -- (m2);
    %
    \node[draw=red,rounded corners,right = 0.2cm of m2] (m3) {$Q,$};
    \node[above = 0.8cm of m3] (l3) {non-empty finite set (set of states)};
    \draw[-,red] (l3) -- (m3);
    %
    \node[draw=red,rounded corners,right = 0.2cm of m3] (m4) {$q_0,$};
    \node[above right of=m4] (l4) {initial state};
    \draw[-,red] (l4) -- (m4);
    %
    \node[draw=red,rounded corners,right = 0.2cm of m4] (m5) {$T$};
    \node[right = 0.1cm of l2] (l5) {state transitions};
    \draw[-,red] (l5) -- (m5);
    %
    \node[right = 0.1cm of m5] (end) {$\rangle$};
}

\end{document}