\documentclass{standalone}
\usepackage{tikz}

\usetikzlibrary{positioning}
\usetikzlibrary{calc}

\begin{document}

% \langle p, \bar{s}, q, \bar{t}, \bar{d}\rangle
% Where:
% p \in Q
% q \in Q
% \bar{s} is a k-tuple representing the current state
% \bar{t} is a k-tuple representing the next state
% \bar{d} is a k-tuple saying what to do for each tape (L,R,S)

\tikz[baseline]{
    \node (start) {$\langle$};
    \node[draw=red,rounded corners,right = 0.1cm of start] (m1) {$p,$};
    \node[text width = 3cm,align=center, above left = 0.1cm of m1] (l1) {
      $p \in Q$\\(current state)};
    \draw[-,red] ($(l1) + (1.2,-0.4)$) -- (m1);
    %
    \node[draw=red,rounded corners,right = 0.2cm of m1] (m2) {$q,$};
    \node[text width = 3cm,align=center,below = 1cm of m2] (l2) {
      $q \in Q$\\(next state)};
    \draw[-,red] (l2) -- (m2);
    %
    \node[draw=red,rounded corners,right = 0.2cm of m2] (m3) {$\bar{s},$};
    \node[above = 1cm of m3] (l3) {a k-tuple representing the current symbol
      on each tape\hspace{3cm}~};
    \draw[-,red] ($(l3) + (-1,-0.2)$) -- (m3);
    %
    \node[draw=red,rounded corners,right = 0.2cm of m3] (m4) {$\bar{t},$};
    \node[above of=m4] (l4) {\hspace{7cm}a k-tuple representing the next symbol
      on each tape};
    \draw[-,red] ($(l4)+(1,-0.2)$) -- (m4);
    %
    \node[draw=red,rounded corners,right = 0.2cm of m4] (m5) {$\bar{d}$};
    \node[below of=m5] (l5) {\hspace{4cm}a k-tuple saying what to do for each tape
      (L,R,S)};
    \draw[-,red] (l5) -- (m5);
    %
    \node[right = 0.1cm of m5] (end) {$\rangle$};
}

\end{document}