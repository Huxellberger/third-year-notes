%\documentclass[a4paper, 12pt]{article}
%\usepackage[top=2.5cm, left=2.5cm, right=2.5cm, bottom=2.5cm]{geometry}
%\usepackage{parskip}
%\usepackage{verbatim}
%\usepackage{listings}
%\usepackage[usenames,dvipsnames,svgnames,table]{xcolor}
%\title{Supertagging}
%\author{}
%\date{}

\lstdefinestyle{xmlish}{
  belowcaptionskip=1\baselineskip,
  breaklines=true,
  xleftmargin=\parindent,
  language=XML,
  showstringspaces=false,
  basicstyle=\footnotesize\ttfamily,
  keywordstyle=\bfseries\color{green!40!black},
  commentstyle=\itshape\color{purple!40!black},
  identifierstyle=\color{blue},
  stringstyle=\color{orange},
}

%\begin{document}
%\maketitle	

Supertagging allows us to handle recursion with regexes. The main principle is
that you write rules but you do not expand them.

Instead you take the first rule that has a match, apply it to all of your text
and see how your text changes. Now, you should ideally have fewer bigger phrases
(the ones which do not have any rules applied to them yet), and some
words/phrases that do have newly added tags around them. Then you can repeat the
process again with the text formed after the first rule has been applied (with
some new tags here and there). If the same rules apply again - you can see how
the recursion happens, if the new rules apply - you can continue parsing the
text.

This resembles divide-and-conquer principle a bit - short strings are easier to
parse, and having multiple rounds of this algorithm allows you to get those
shorter strings. (This is where the \textbf{islands} bit comes from, correct me
if I am wrong!)

This thing is in-place - you are re-writing your entire text as you apply those
regexes to it, so you have to get it right the first time.

Here is how those rules look like:

\begin{verbatim}
	ctags = { 'nn': choice(tag'NN.'),
						   choice(tag('AJ.'), tag('noun'))+tag('nn')),
		      'np0': tag('(AT|D).')+tag('nn'),
	          'np': choice(tag('np0'), tag('np')+tag('pp')),
	          'pp': tag('PR.')+tag('np')}
\end{verbatim}

In this rule we can see that \texttt{nn} is defined in terms of itself (the
\texttt{tag('nn')} bit at the end). Where we have \texttt{choice(\ldots)} - this
is the same as having an \texttt{OR} operator. So the first rule would look like
this \texttt{(NN|AJ|(noun + nn))}. This will apply to simple little phrases that
have just a noun, or just an adjective, or a noun followed by a noun.

The next rule means we have a determiner (an article or otherwise) followed by a
noun. And so forth.

Below is a dry run of this set of rules (taken from the course notes along with
the rules). Can you work out what rules have been applied at each stage and how
text has changed as a result?

\textbf{Initial (before any rules)}
\lstset{escapechar=@,style=xmlish}
\begin{lstlisting}
<AT0>the</AT0>
<AJ0>fat</AJ0>
<AJ0>old</AJ0>
<NN1>man</NN1>
<PRP>with</PRP>
<AT0>a</AT0>
<AJ0>big</AJ0>
<NN1>nose</NN1>
\end{lstlisting}

\textbf{After the first pass}
\lstset{escapechar=@,style=xmlish}
\begin{lstlisting}
<AT0>the</AT0>
<AJ0>fat</AJ0>
<AJ0>old</AJ0>
<nn>
	<NN1>man</NN1>
</nn>
<PRP>with</PRP>
<AT0>a</AT0>
<AJ0>big</AJ0>
<nn>
	<NN1>nose</NN1>
</nn>
\end{lstlisting}

\textbf{After the second pass}

\lstset{escapechar=@,style=xmlish}
\begin{lstlisting}
<AT0>the</AT0>
<AJ0>fat</AJ0>
<nn>
	<AJ0>old</AJ0>
	<nn>
		<NN1>man</NN1>
	</nn>
</nn>
<PRP>with</PRP>
<AT0>a</AT0>
<nn>
	<AJ0>big</AJ0>
	<nn>
		<NN1>nose</NN1>
	</nn>
</nn>	
\end{lstlisting}

\textbf{After the third pass}

\lstset{escapechar=@,style=xmlish}
\begin{lstlisting}
<AT0>the</AT0>
<AJ0>fat</AJ0>
<nn>
	<AJ0>old</AJ0>
	<nn>
		<NN1>man</NN1>
	</nn>
</nn>
<PRP>with</PRP>
<np0>
	<AT0>a</AT0>
	<nn>
		<AJ0>big</AJ0>
		<nn>
			<NN1>nose</NN1>
		</nn>
	</nn>
</np0>
\end{lstlisting}

\textbf{After the fourth pass}
\lstset{escapechar=@,style=xmlish}
\begin{lstlisting}
<AT0>the</AT0>
<AJ0>fat</AJ0>
<nn>
	<AJ0>old</AJ0>
	<nn>
		<NN1>man</NN1>
	</nn>
</nn>
<PRP>with</PRP>
<np>
	<np0>
		<AT0>a</AT0>
		<nn>
			<AJ0>big</AJ0>
			<nn>
				<NN1>nose</NN1>
			</nn>
		</nn>
	</np0>
</np>

\end{lstlisting}

\ldots \\
\textbf{After the final pass}
\lstset{escapechar=@,style=xmlish}
\begin{lstlisting}
<np>
	<np>
		<np0>
			<AT0>the</AT0>
			<nn>
				<AJ0>fat</AJ0>
				<nn>
					<AJ0>old</AJ0>
					<nn>
						<NN1>man</NN1>
					</nn>
				</nn>
			</nn>
		</np0>
	</np>
	<pp>
		<PRP>with</PRP>
		<np>
			<np0>
				<AT0>a</AT0>
				<nn>
					<AJ0>big</AJ0>
					<nn>
						<NN1>nose</NN1>
					</nn>
				</nn>
			</np0>
		</np>
	</pp>
</np>
\end{lstlisting}
(for the complete example see the course notes)