The first few slides (under the big Machine Translation title, p565-566) just
explain the logic of the English and Arabic sentences - ``The man wrote a book''
and ``ktb Alrjl ktb'' which is the same thing in Arabic with some vowels
omitted.

Then he goes on to talk about how would we represent the meaning. (p568)

The [., (arg (claim)] bits (p571) are again the translation rules, Allan drew
the two different parse trees for English and Arabic (will be included as a
separate picture, [number 1]).

So those rules just say to translate this sentence ``The man wrote a book'' into
Arabic.

OK, by inspecting the tree structure you probably would have guessed that if you
literally translate from English into Arabic and the other way, speak like Yoda
you will. ``Wrote the man book'' would be the literal translation of ``yktb
Alrjl ktb'' (Arabic $\rightarrow$ English, literal translation, I will trust my
friend Google Translate on that one). There are a few other catches to it.
Arabic has the no indefinite article ``a''. Whoops. And also... The definite
article ``The'' is attached to the noun. Instead of Al rajul (The man) they have
``Alrajul''. A bit like English has sometimes ``doesn't'' or ``o'clock'' written
instead of ``does not'', ``on the clock''(is that even the correct version of
o'clock?).

But that means the Arabic tree messes up, because it doesn't correspond
correspond to the English one (No ``a'', ``the man'' as one word...). So he
discusses what would he do to still make those trees align in the machine
translation. (That's how I understand it, yeah.) He could treat ``Al'' as
separate word for analysis, and then attach it on afterwards. But he will have
to delete the indefinite article ``the'' to translate it into Arabic.

\textbf{Transfer rules}

p.575 This is where the actual transfer rule part starts (sort of). [-def] means
the indefinite article. [+def] means the definite article. This, I am guessing
is a representation of a sentence in a tree in a flat form (I am guessing,
again, he didn't say this on this specific part but, the ``.'' is at the top,
because he mentioned that the meaning of the sentence changes with punctuation
in some of the earlier lectures). Then we have the verb and other things with
articles attached (English: book has ``a'', man has ``the''. Arabic doesn't, but
we sort of map them anyway - we know that ``book'' has ``a'' in English. In
Arabic it doesn't, but that's where the ``a'' would go if it was present in
Arabic).

p.576 we see the syntax. Again, I will do it as a separate drawing - see [number
2] (because I am a visual person :) ). And apply this recursively to all of the
English tree to get an Arabic one.

Start with applying all the little rules to all of the tree (word->word
mappings, as far as I understood), then apply bigger rules (such as this one? I
may be wrong in that specific part. He just said the bits not in brackets.)

p.578 has the same thing with an axillary. The thing is, English has more tenses
than Arabic. So if you translate:

1) English: ``I am writing a book''
2) Arabic(translated): ``I write a book''
3) English (translated from Arabic): ``I write a book''

You see? The original meaning is lost! No problem when you translate TO Arabic.
But when you translate FROM Arabic... This is what he means in  p.580. The rules
themselves though, allow you to convert sentence to tree and backwards - tree to
sentence.

p.581 He carries on with the point he made earlier - what if one language has
distinctions the other one doesn't. He gives you the example of ``they''.
Russian also has a single, gender-neutral form ``oni'', so I was puzzled to find
about those distinctions. French has ils and elles for plural masculine and
plural feminine forms, Arabic, as well as having those two gender distinctions
also distinguishes whether there are two of them or more than two.

So again, when you have to translate - part of the meaning can get lost:

1) John and Mary went to the cinema. They enjoyed the picture. Here we know from
the surrounding context that we have a male and a female. So to translate that
we use plural masculine (because if at least one person is a male - masculine
form is used. This is true in all the languages I studied that make that
distinction), representing 2 people for Arabic.

2) Sam and Alex went outside. They enjoyed the sun. Here two gender-neutral
names are used. How the heck are you supposed to know what form of ``they'' to
use now? Only if you have more surrounding context, if you don't - Houston, we
have a problem!

Also, earlier tense distinctions still apply here.
(He is so correct about the bare plural!!!)

p.584 Here he explains how some things don't map to other languages as a single
word. ``I am looking for a unicorn'' translates to French as ``I am in search of
a unicorn''

``looking for'' (2 words) $\rightarrow$ ``in search of'' (3 words)

The opposite is true for English. You can say ``We dined'' but you don't say
``We breakfasted'', ``We tead'' (as opposed to ``we had breakfast'', ``we had
tea'') and you don't really say ``We lunched''. Some languages have (Russian
definitely has) them all as verbs.

For idioms, even if an idiom with an equivalent meaning exists you still
translate it as a bare meaning (at least according to Allan, I clarified).

So if you have ``kick the bucket'' in English and say you want to translate it
to Russian, instead of using the equivalent phrase (``kon'ki otbrosit'',
literally translates as ``throw away the ice skates'') you translate it as
``become dead''.

\textbf{The MT pyramid} (p.586)

The gist here is the fact that no matter what language you use (out of the ones
we studied) you have the same parts of speech. You say ``I'', ``je'', ``ich'',
``ya'', ``watashi'' or ``he'', ``il'', ``er'', ``on'', ``kare''... you know they
are pronouns. Doesn't matter what language, it's a pronoun! That's an
interlingua.

Another example of interlingua: transitive verb with subject and object becomes
a transitive verb with subject and object.

Interlingua is very general.


p.591 (as far as I understood from there)  Textual entailment vs machine
translation: in textual entailment you use logic to figure out the links:

John and Mary got divorced --> John and Mary were once married.

Machine translation - you map from one language to another, preserving the gist.
Text simplification can also be interpreted as translation (Wikipedia:
English--> Simple English!)

\textbf{Ambiguity in MT} (p.592)

3 kinds of ambiguity:

structural ambiguity:

strawberry jam jar

Ok, that's pretty obvious. It's a jar of strawberry jam! Think again. They are 3
nouns. Machine can't just say what you just said. Machine needs to know whether
it is

(strawberry jam) jar or strawberry (jam jar)

Basically which two nouns to group to get the right meaning. For humans it is
not such a big deal. Other languages (Dutch):

aardbeienjam jar

The words that need to be ``bracketed'' are just mashed together!

Scope ambiguity:

This one was trickier (at least for me). So I suggest you watch this: video:
\url{https://www.youtube.com/watch?v=XC-MGuj75zQ} This guy will probably explain
better than me. The next few lines are essentialy a TL;DR of the video.

Basically we use quantifiers in our speech. They are not obvious, because our
mind works differently from a machine.

Any, every, a...

These guys map to ``there exists'' and ``for all''. UBUQUITOUS LOGIC!

All John's friends went to a fantasy land.

There are two meanings:

1) ``All John's friends went to the same fantasy land.''
2) ``All John's friends went to different fantasy lands, it's just for every friend there exists a fantasy land they went to - Sarah went to Oz, Stephen went to Narnia''

The first meaning comes from ``all'' friends (``all'' winning over ``a''), that
is for each friend there exists a single one fantasy land that is the same for
everyone.

The second meaning comes from ``a'' (``a'' winning over ``all'') - for each
friend there exists a fantasy land, which does not have to be the same fantasy
land.

Then he goes on to explain how does our brain pick the first interpretation,
even though the second one is more logical.

TL;DR ends.

The 2 examples Allan gives in the lectures:

1) You can't invite John. He will drink EVERYTHING.
2) You can't invite John. He will drink ANYTHING.

Here Allan explains that these two are both universal quantifiers but have a
different scope. That is (WARNING: over-complication coming through):

1) John will drink EVERYTHING.
There exists ONE situation where he will drink ALL the drinks.
2) John will drink ANYTHING.
Whatever you offer John, there exists at least one situation where he will drink
it. Total: MANY situations, many possibilities.

And the unicorn example. This one was not obvious to me.

The ambiguity comes from whether there actually exists a unicorn John is looking
for or does it not exist? This can be interpreted in two ways:

1) John is looking for a unicorn. Don't tell him they don't exist.
There ISN'T a unicorn John is looking for.
2) John is looking for a unicorn. It ran off the castle this morning.
There IS a unicorn John is looking for.

Lexical ambiguity: When you have a word with several meanings and you don't know
which one is intended.

So, from language to language structural ambiguity (not always, but sometimes)
and scope ambiguities transfer. Lexical ambiguity doesn't.

\textbf{Bilingual corpora, parallel corpora} (p.597)

Here he just talks about how can he literally compare words side by side with
some adjustments. He will match two words if they are in the same
position/roughly the same number of consonants/all this bunch of rules
together/etc.


\textbf{Alignment} p.604

Because of the word order he will have to use a thing similar to string-edit
distance to be able to do that - insert a word on one language, delete on
another one.

professor of formal linguistics --> professeur de linguistique formelle.

Insert ``formal'', delete ``formelle''.

Then he goes on to talk about what happens if he deletes all the words that
match with their positions. He will get some unmatched ones and he can
reasonably match the unmatched ones by, for example, looking at their lengths.

And that's the bit that I got to. :) 
To be continued?