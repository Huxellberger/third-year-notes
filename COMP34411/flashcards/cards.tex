\card { Noun } {
	A person or a thing. E.g. `dog', `house', `Todd', `idea'.
}

\card { How are English nouns typically pluralised? } {
	Adding `s' to the word.
}

\card { Verb } {
	A state, action, or event. E.g. `sleep', `live', `kiss', `run'.
	\\\vspace{0.5em}
	Verbs often carry markers to indicate whether an event is ongoing, has happened, or
	maybe to further describe it (e.g. Todd lived there.).
}

\flashcard {
	% TODO: Check this one
	`\blank{am}' is the present tense form of `be'.
}

\flashcard {
	`\blank{was}' is the past tense form of `is'.
}

\card { Infinitive } {
	Infinitive = `To' + verb.
	\\\vspace{0.5em}
	E.g. `to smash', `to eat', `to write'
}

\card { Complete sentence } {
	Begins with a capital letter, finishes with an end mark (period, question mark etc), and contains at
	least one main clause.
}

\card { Main clause } {
	Contains a subject and a verb, experssing a complete thought. E.g. `Todd wrote flashcards.'
	\\\vspace{0.5em}
	`Todd' is the subject, `wrote' is the verb.
}

\card { Participle } {
	There are two varieties; past and present, and they're used to transform verbs.
	\\\vspace{0.5em}
	E.g Giggle $\rightarrow$ giggled / giggling, bring $\rightarrow$ brought, bringing.
}

\flashcard { Present participles always end in \blank{ing}. }

\card { Adjective } {
	Describes a property, usually adding information to a noun e.g. `green grass'.
	\\\vspace{0.5em}
	Can be suffixed with `er' or `est' to give relative scale e.g. `greenest grass'.
}

\card { Adverb } {
	Describes a property of a verb. E.g. `He revised frantically'.
}

\card { Pronoun } {
	A word used to denote some thing / person that is identifiable from the context of the sentence.
	\\\vspace{0.5em}
	E.g. we, I, me, you, him, it etc.
}

\card { Preposition } {
	A word that provides more information to a noun / verb by linking it to another entity.
	\\\vspace{0.5em}
	E.g. `The programmer with four hands works fast' (with)
}

\card { Auxiliary } {
	Provides information about when an event happened.
	\\\vspace{0.5em}
	E.g. `I am sleeping', `I have slept', `I might sleep'.
}

\card { Determiner } {
	Tells you how to use a description of an entity. E.g:
	\\\vspace{0.5em} `The fat person' $\rightarrow$ This specific fat person.
	\\\vspace{0.5em} `A fat person' $\rightarrow$ Any fat person.
	\\\vspace{0.5em} `Most fat people' $\rightarrow$ Something about most fat people.
}

\card { Person (first, second, third) } {
	First $\rightarrow$ Any group including the speaker; I, me, we, us.
	\\\vspace{0.5em}
	Second $\rightarrow$ Any group including the hearer; you (also old things like thee, thou).
	\\\vspace{0.5em}
	Third $\rightarrow$ Anything else; he, she, they, her, him, it
}

\card { Number (singular / plural) } {
	Singular $\rightarrow$ One thing e.g. cat
	\\\vspace{0.5em}
	Plural $\rightarrow$ Many things, e.g. cats
}

\card { Tense } {
	Tells you when something happened (past, present, future).
	\\\vspace{0.5em} Past $\rightarrow$ He ate the burger.
	\\\vspace{0.5em} Present $\rightarrow$ He is eating the burger.
	\\\vspace{0.5em} Future $\rightarrow$ He will eat the burger.
}

\card { Aspect } {
	Tells you how the event was related to the tense. E.g:
	\\\vspace{0.5em} He crossed the road $\rightarrow$ He made it all the way over the road
	\\\vspace{0.5em} He was crossing the road $\rightarrow$ He was part way through crossing the road
}

\card { Voice (passive / active) } {
	Whether you did something (active) or whether it was done to you (passive). E.g.
	\\\vspace{0.5em} Active $\rightarrow$ I made the flashcards.
	\\\vspace{0.5em} Passive $\rightarrow$ I was given the flashcards.
}