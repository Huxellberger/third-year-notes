\section*{Introduction}

Enabling computers to use `natural language' (the kind of language that people
use to communicate with one another) is becoming more and more important. It
allows people to communicate with them without having to use strange artificial
languages and awkward devices like keyboards and mice; and it allows the
computer to access the enormous amount of material that is stored as natural
language text on the web.

This course provides an introduction to this area, mixing theory (if you don't
understand the theory of how language works you cannot possibly write programs
that understand it) with practice (if you haven't written or played with tools
that embody the theory, you can't get a concrete handle on what the theory
means).

\section*{Aims}

The course unit aims to teach the techniques required to extend the theoretical
principles of computational linguistics to applications in a number of critical
areas.

\begin{itemize}
  \item To demonstrate how the essential components of practical NLP systems
  are built and modified.
  \item To introduce the principal applications of NLP, including information
  retrieval \& extraction, spoken language access to software services, and
  machine translation.
  \item To explain the major challenges in processing large-scale, real-world
  natural language.
  \item To explain the principles underlying speech recognition and synthesis,
  and to explore the power of `black box' tools for these tasks.
  \item To give students an understanding of the issues involved in evaluating
  NLP systems.
\end{itemize}

\section*{Attribution}

These notes are based of the course notes by Allan Ramsey (found at
\url{http://studentnet.cs.manchester.ac.uk/ugt/2015/COMP34411/COMP34411.pdf}).
Thanks Allan for providing such thorough slides! I've probably missed out bits
and misinterpreted other bits of those, so you shouldn't just use my notes for
revision! Thanks also to Tatjana Sidorenko for donating her supertagging and
machine translation notes so that this document covers the whole course.

\section*{Contribution}

Pull requests are very welcome:
\url{https://github.com/Todd-Davies/third-year-notes}
