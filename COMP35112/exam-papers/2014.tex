\documentclass{report}

\usepackage[margin=1in]{geometry}
\usepackage{tabularx}

\begin{document}

\begin{enumerate}
\item
  \begin{enumerate}
  \item Cache coherence is a key feature in chip multiprocessor design
    because it allows multiple cores to have the same view of
    memory. This is important since it allows for the maximum amount
    of coordination and collaboration between cores, which in turn,
    lets programmers obtain a maximal performance from parallelising
    programs.

    Cache coherency is important because it avoids the need for each
    write to memory to be written \textit{all the way} to memory
    (which can take many hundreds of CPU cycles in the worst case),
    but instead lets memory be written to a fast cache (which can take
    one cycle, or only a few cycles) and written later, while
    maintaining the semantics of the computation on the accessing core
    and other cores. This also applies when reading values from
    memory.
  \item Modified means that a core has written to the memory location,
    but the write has not been propagated through to main memory.

    Shared means that this cache holds the up-to-date value of a
    memory location, and at least one other cache holds the same
    value.

    Invalid means that the value held for the memory location has been
    updated by either another CPU core or a DMA operation since the
    memory location was loaded.
  \item
    \begin{enumerate}
      \item This is what the MESI protocol does. The advantage of
        having the `exclusive' state is that the core does not need to
        send out an invalidate broadcast on the system bus if it
        writes to the value.
      \item This is an optimisation since one core could evict the
        cache line from its cache and not have to write the line back
        to memory, since the other cache still retains the updated
        value, and this cache would go to state M or O depending on if
        there was more than one core sharing the value. This is the
        MOSEI protocol (Owned).
    \end{enumerate}
  \item
    \begin{tabularx}{\textwidth}{llX}
      Cycle & States & Explanation\\ \hline
      1.    & EII    & Core one has loaded x and is the only core to
      have done so\\
      2.    & SSI    & Both core one and two have loaded x; they're
      both in the shared state\\
      3.    & SSI    & Add does nothing to memory\\
      4.    & SSI    & Add does nothing to memory\\
      5.    & MII    & Core one writes to the cache, and invalidates
      core two.\\
      6.    & OIS    & Core three loads the value, and receives it
      directly from core one, which becomes an owner\\
      7.    & OIS    & Add does nothing\\
      8.    & IIM    & Core three's write invalidates core one. Core
      one may write back to memory now if this is not optimised out.\\
      9.    & IMI    & Core two writes its value to memory, and
      invalidates core three, which may write back.\\
    \end{tabularx}
  \item Since the threads do not synchronise their memory accesses,
    the actual order that their instructions are executed is not
    defined at runtime. If two threads read the same value from
    memory, they then both add one to it, and then write back one
    after the other, then the location will only be incremented
    once. The memory location could be incremented by either one two
    or three. (Include example run in exam for each case)
  \item Transactional memory would require bits to indicate whether
    the cache line is in the readset or writeset of the
    transaction. This can be done with two extra bits (one for each
    set). These bits are then used at the end of transactions
    (assuming lazy conflict detection) to determine whether it is safe
    to commit the transaction.
  \end{enumerate}
\item
  \begin{enumerate}
  \item Synchronisation constructs are required in Java since it is a
    high level language, and single statements compile down to many
    machine instructions. As a result, these statements do not happen
    atomically at runtime, for example, incrementing an integer would
    be a RMW (read, modify, write) operation.

    \begin{verbatim}
      LOAD R1, x
      ADD R1, R1, #1
      STORE R1, x
    \end{verbatim}

    If two threads were to execute this code at the same time, then
    they could both read, both increment their own registers, and then
    both write. The result would be that the memory location would be
    incremented once, not twice.

    Synchrnoisation constructs in Java let the compiler know to use
    atomic operations such as load-link and store conditional.
  \item Blah
  \item Long pipelines make CISC instructions slooow.
  \item
    \begin{verbatim}
      loop LDR R1, semaphore
           CMP R1, #1
           BNE loop
           LDL R1, semaphore
           CMP R1, #1
           BNE loop
           MOV R1, #1
           STC R1, semaphore
           CMP R1, #1
           BNE loop
           MOV PC, LR
    \end{verbatim}
  \item The first time that STC happens, the load link flag will be
    cleared, so that the second time it happens, the load link flag
    will not be set and the whole thing will reset. In short, you
    cannot nest load linked instrucitons.
  \item Obtain one lock, then spin wait to obtain the other. Problem;
    you might get deadlock if two threads wait on each other.
  \end{enumerate}
\item This question is okay.
\item
  \begin{enumerate}
  \item Main architecutral features are:
    \begin{itemize}
    \item GPGPU's have a massively parallel architecture that utalises
      Streaming Multiprocessors (SM's) to run SIMD style
      computations. GPU's also have their own on-board memory which
      makes up a completely seperate address space from that of the
      rest of the system. In order to do useful computation, data is
      transferred over a fast (and wide) bus to the GPU where it is
      stored in GPU memory (usually facilitated by DMA).

      Each SM has multiple `cores' (or `CUDA cores' using NVIDIA's
      terminology), each of which is a relatively simple processor
      with a small amount of cache, a short pipeline and a reduced
      instruction set (very RISC). Most of the die area on a GPU core
      is dedicated to the ALU's rather than control logic or cache,
      since latency is sacrificed for throughput in GPU computations.
    \item CUDA and OpenCL build on the existing language to let
      programmers provide hints of how to parallelise, while mostly
      leaving the low level details to the framework (thought with
      both libraries you can be specific about how parallelisation is
      done).

      For example, with CUDA, you write kernels which are executed on
      the GPU device in a massively parallel manner. Each kernel is
      initialised on multiple threads, and the number of threads over
      which the kernel is started is defined using the `dim3'
      construct. A grid size and a block size is given (the total
      number of threads launched is equal to the size of the grid
      multiplied by the size of each block in the grid). For example:

      \begin{verbatim}
        __global__ myKernel(int* array) {
          int id = threadId.x + (blockDim.x * blockId.x);
          array[id] = array[id] + 1;
        }

        ...

        int* h_array, d_array;
        h_array = calloc(sizeof(int) * 1000);
        cudaMalloc(&d_array, sizeof(int) * 1000);
        cudaMemcpy(d_array, h_array,
          sizeof(int) * 1000, cudaMemcpyHostToDevice);
        myKernel<<<dim3(10,1,1), dim3(10, 10, 1)>>>(d_array);
        cudaMemcpy(h_array, d_array,
          sizeof(int) * 1000, cudaMemcpyDeviceToHost);
        cudaFree(d_array);
      \end{verbatim}
    \end{itemize}
  \end{enumerate}
\end{enumerate}         

\end{document}
