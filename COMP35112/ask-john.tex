------
GPU question
c flops, g flops, b bytes, nxn matrices

Everything in the host, then n^3 floating point multiply-adds. Therefore n^3 flops. Therefore time is (n^3)/c.

Everything in device. 2 x n^2 x 8 (doubles) bytes to transfer, to get time is (2 x n^2 x 8) / b.
Computation is (n^3)/g.
Transfer back is (n^2 x 8) / b
All is (24 x n^2) / b + n^3 / g

Therefore for GPU to be faster:

((24 x n^2) / b) + (n^3 / g) < n^3 / c

Solve for n.
------
TLS

Speculative threads need to keep a copy of all the things that
non-speculative threads have accessed (read and write sets).

Four states; not accessed (N), speculatively loaded (S), modified (M),
speculatively loaded and modified (SM)
------
Stages of a transaction

Start (copy data)
Run (update local data)
Commit/Abort (work out if it's okay to commit, then copy data back)
------


- Slide 18, lecture 7: Why is the load linked flag removed if the
thread is descheduled?

So that you can assume a clean state when the thread resumes, since
the descheduling takes so long anyway. Load link flag wouldn't carry
on working during the time that the thread was descheduled.

- Slide 10, lecture 11: TM wouldn't guarantee that you solve the
problem first time though, right? There may be cases where there are
no paths and you need to remove routes (i.e. backtrack)?

- Lecture 14 (GPU Lecture): Is the old architecture of the GPU (pre
CUDA) relevant for the exam?

Naah

- Exam questions (technique): on past papers, I find myself putting
loads down for the first part of the question, and then repeating
myself later. Any advice?

- 2014 1c) ii); Go over the answer here; what happens to each code?



- 2014 1d): Check this is correct. MOSEI is a little unclear in the notes



- 2014 1f): This seems a little short for four marks, but I can't
think of what else to put

Three bits; part of transaction, read set, write set


TODO: Revise lecture 13 (memory consistency)
