% Set the author and title of the compiled pdf
\hypersetup{
  pdftitle = {\Title},
  pdfauthor = {\Author}
}

\section{The need for parallelism}

Even though we've been unable to increase the clock speed of processors since
around 2005, we have seen the `power' of processors roughly double every 18-24
months since then in line with Moore's Law. The reason why, is that we have been
able to increase the amount of transistors in chips (due to the feature size
decreasing), and use the extra ones to provide more processing cores, which are
able to process data in parallel. The degree of parallelism is increasing as
time progresses.

In an ideal world providing a greater degree of parallelism would merely entail
chip designers copy and pasting multiple processor cores onto the silicon, and
programmers getting linear performance increases. In practice, there are lots of
architectural issues (such as how processors are connected and how they're
organised) as well as software issues (how do we make our app run on multiple
cores).

As transistors are becoming smaller, we can also make them switch faster. The
switching speed is determined by $R*C$, where $R$ is the resistance and $C$ is
the capacitance. When we reduce the area of the transistor, $C$ decreases, so in
doing so, we make the circuit able to compute faster.

This was fine until 2005, at which point we started to have problems with things
like \textit{interconnect capacitance} (the capacitance between neighbouring
wires). Also, the power density increased (smaller transistors means more per
unit area), which has serious cooling implications. As we approach the
theoretical limit of one atom per transistor, any impurity in the silicon
becomes a major issue.

We have tried using extra transistors to build more complex single core
processors (using Instruction Level Parallelism (ILP)) and by adding bigger
caches so that they exhibit lower miss rates, however both of these techniques
suffer from diminishing returns. Control statements such as \texttt{branches}
make us have to periodically throw away all the partially completed instructions
in a pipeline, and caches already have hit-rates in the high $90$'s.

Though we might be able to increase the number of cores on a chip, how does a
programmer utilise this extra power? It is relatively easy for an operating
system to schedule programs so that they can run on different cores and
therefore have true multi-tasking (process level parallelism), but what if we
want to make one program run faster by running it over many cores?

\subsection{ILP vs TLP}

Instruction Level Parallelism and Thread Level Parallelism are two different
approaches to utilising parallel hardware, and both can be used at the same
time.

In ILP, the processor is able to execute instructions out of order and in
parallel, meaning that fewer clock cycles are needed to execute the same number
of instructions. This form of parallelism is very limited, and can only be used
in certain situations. Vector parallelism is similar, and lets you do things
like do four 8-bit additions in one instruction (by splitting a 32-bit word into
four 8-bit parts).

In TLP, a program can be composed of separate threads, each
being its own sequence of instructions. Many threads can be executing in
parallel and since their instructions are independent of each other, can be
interleaved on the processor (or run on multiple cores). The output of threaded
programs must be deterministic, i.e. for all possible sequences of execution,
the output must be the same.

TLP is far more general purpose than ILP (and much more so than vector
parallelism, which is only really useful in simple operations like array
addition). However, it relies on the programmer finding a way to express an
algorithm in a parallel manner.

\subsection{Data and instruction parallelism}

\textit{Flynn} classified parallelism as \textbf{instruction stream} and
\textbf{data stream} parallelism.

Data parallelism is when the same computation is carried out on multiple
elements of some dataset, usually an array. Vector parallelism is an example of
this. Only certain problems amenable to this type of parallelism can be sped up
in this manner.

Instruction parallelism is when multiple instructions can be executed in
parallel (with no side effects that cause problems to each other).

Different combinations of these types of parallelism have different names:

\begin{description}
  \item \textbf{SISD}: Single Instruction Single Data is like a normal program.
  A serial sequence of executions working on a stream of data, one word at a
  time.
  \item \textbf{MIMD}: Multiple Instruction Multiple Data is the most parallel
  one, where there are multiple instruction streams and they can all operate on 
  their own independent stream of data. This is most normal computer chips.
  \item \textbf{SIMD}: Single Instruction Multiple Data is stuff like when you 
  use instructions like \texttt{ADD8} to process multiple data elements at once 
  with the same instruction, or a GPU processing lots of pixels at once with 
  the same filter. If it's not a vector instruction, then it's one or more 
  processors working in lockstep to process elements of an array or something.
  \item \textit{SPMD}: Single Program Multiple Data is a generalisation of SIMD,
  where different processors execute the same code but don't need to be in 
  lockstep. This is most often how parallel programs are written for CPU's.
\end{description}

\subsection{Connecting processors}

In order to make them work in parallel, multiple cores of a processor need to be
connected to each other, and to memory. There are lots of different ways to do
this, and the best way depends on the use case.

\begin{description}
  \item The processors can be laid out in a grid. In this case, each processor
  can communicate with its neighbours, and memory is usually private to each 
  core. In order for cores to access memory in other cores, they must send 
  messages through the grid.
  \item A Torus is a variation of the grid, where each edge is linked to the 
  opposite edge. This makes a doughnut shape (in a logical, not physical sense)
  and means that fewer steps are needed to communicate between cores.
  \item A bus can be used to connect multiple cores. The memory is usually
  situated on the bus, and all cores have access to it (though they may also
  have their own memory instead). Time slicing is used to give equal access to 
  the bus, but can make the bus become a processing bottleneck.
\end{description}

There are, of course, more types of interconnects. Crossbars are where each node
is connected to each other node (which has a complexity of $n(n-1)$), but the
best ones are usually tree structures or hierarchical buses where the complexity
is logarithmic ($n log(n)$).

\subsection{Shared and distributed memory}

Shared memory is accessible from all of the cores and every part of the
computation, while distributed memory is spread out in different components, and
is usually only accessible by the component that owns it. We are considering
systems that are either one of these, or the other. Either one of these can
emulate the other from a software point of view; it is fairly easy to provide
abstraction layers that make a distributed memory behave like a shared one, or
impose restrictions on shared memory so that parts are unavailable to certain
components. Imposing a foreign memory layout onto the hardware comes at a
performance penalty.

Most supercomputers use distributed memory, since its easier to build, provides
a higher total communication bandwidth and is more suited to many data-parallel
problems. However, programs using data sharing (shared memory) are widely seen
to be easier to code than programs using message passing (distributed memory),
so there is an overhead involved with distributed memory.
