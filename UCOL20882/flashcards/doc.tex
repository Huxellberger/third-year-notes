\documentclass[frontgrid]{flacards}
\usepackage{color}
% For funky database symbols
\usepackage{newlfont}
\usepackage{tabularx}
\usepackage{graphicx}
\usepackage{mathtools}

\definecolor{light-gray}{gray}{0.75}

\newcommand{\frontcard}[1]{\textcolor{light-gray}{\colorbox{light-gray}{$#1$}}}
\newcommand{\backcard}[1]{#1} 

\newcommand{\flashcard}[1]{% create new command for cards with blanks
    \card{% call the original \card command with twice the same argument (#1)
        \let\blank\frontcard% but let \blank behave like \frontcard the first time
        #1
    }{%
        \let\blank\backcard% and like \backcard the second time
        #1
    }%
}

\begin{document}

\pagesetup{2}{4} 

\flashcard{Each DNA molecule is packed into a \blank{chromosome}.}

\flashcard{\blank{Genes} contain instructions for making \blank{proteins}.}

\flashcard{The two strands of DNA twist to form a \blank{double helix}.}

\flashcard{When replicating, the \blank{hydrogen bonds} between the DNA strands break, and \blank{new bases} come to bind with the exposed ones on the separated strands to form new strands.}

\flashcard{Proteins act alone or in \blank{complexes} to perform many cellular functions.}

\card{The four DNA bases are...}{Adenine, Thymine, Guanine, Cytosine}

\flashcard{A \blank{sugar-phosphate} backbone provides structure for the DNA.}

\flashcard{\blank{Hydrogen} bonds hold the two strands of DNA together.}

\flashcard{\blank{Adenine} binds to \blank{Thymine}, \blank{Cytosine} binds to \blank{Guanine.}}

\flashcard{Before a cell divides, its DNA is duplicated using \blank{semi-conservative replication}.}

\card{What is the Karyotype?}{The 23 pairs of chromosomes in the cell.}

\card{What is an autosome?}{One of the 22 pairs of normal chromosomes in humans.}

\card{In addition to the autosomes, what other chromosomes are there?}{One pair of sex chromosomes.}

\flashcard{\blank{Meiosis} is the process where a sperm producing cell or an egg producing cell makes a new cell with 23 chromosomes.}

\flashcard{\blank{Mitosis} is when an exact replica of the genome is made (46 chromosomes).}

\flashcard{\blank{Meiosis} is when only one chromosome from each pair is passed on to the new \blank{gamete} (sperm/egg).}

\flashcard{DNA $\xrightarrow[]{\blank{transcription}}$ RNA $\xrightarrow[]{\blank{translation}}$ protein}

\flashcard{When a gene is \blank{transcribed}, it forms many \blank{RNA} molecules.}

\flashcard{\blank{RNA} molecules get \blank{translated} into proteins.}

\card{Define an allele}{Any of several forms of a gene, usually arising through mutation. Alleles are responsible for hereditary variation.}

\card{Define polymorphism (in the context of DNA)}{The existence of several alleles for one gene locus. Individuals have one or two alleles per locus.}

\flashcard{\blank{Homozygous} is when a person has two copies of one allele on a gene locus.}

\flashcard{\blank{Heterozygous} is when a person has two different alleles on a gene locus.}

\flashcard{A gene is \blank{recessive} if the \blank{mutated} protein that it produces can be compensated for by the correct protein produced by \blank{an alternative allele}.}

\flashcard{If a mutated gene produces proteins that fulfil a new function, then it may be \blank{co-dominant}, since the original function will be fulfilled by \blank{the other allele}.}

\flashcard{Genes can be \blank{recessive}, \blank{dominant} or \blank{co-dominant}.}

\card{Define genotype.}{The genetic make-up of an individual, which includes the genes or alleles present in it.}

\card{Define phenotype}{The physical appearance of an individual, including its observable or measurable traits.}

\flashcard{The phenotype is controlled by \blank{proteins} derived from \blank{genes}, and the \blank{environment}.}

\card{What bloodgroup is made from two co-dominant alleles?}{AB}

\flashcard{
  Blood groups:
  \begin{tabular}{>{$}c<{$}|>{$}c<{$}>{$}c<{$}>{$}c<{$}}
        & I^A & I^B & i\\ \hline
    I^A & \blank{A}  & \blank{AB}  & \blank{A}\\
    I^B & \blank{AB} & \blank{B}   & \blank{B}\\
    i   & \blank{A}  & \blank{B}   & \blank{O}
  \end{tabular}
}

\flashcard{Allele frequency is linked to \blank{the fitness it provides} to its \blank{carriers} in a given \blank{environment}.}

\card{Define genetic fitness}{The reproductive success of a genotype, measured as the number of offspring produced by and individual that survive to a reproductive age relative to the average age for the population.}

\flashcard{If an allele provides \blank{an advantage}, it is likely to \blank{persist} and become \blank{more prominent} in a given population.}

\flashcard{Mutations have allowed us to \blank{diversify} our diet. This includes a mutation that lets us produce \blank{lactase} during adulthood (to drink milk) and another one that reduces the function of a \blank{bitter substance taste receptor} allowing us to eat broccoli and sprouts! This is an example of \blank{natural selection}.}

\flashcard{Carriers of \blank{sickle cell anaemia} alleles are \blank{asymptomatic} and get protection from malaria.}

\flashcard{Carriers of \blank{sickle cell anaemia} alleles die if they are \blank{homozygous} since their haemoglobin does not function well.}

\flashcard{People \blank{homozygous} for a mutation affecting \blank{CCR5} are asymptomatic and immune to HIV. Probably because this gave protection against \blank{the plague} and \blank{smallpox} in the past. This mutation is less effective against pathogens from \blank{developing countries}.}

\flashcard{Environment interaction can influence the genotype. \blank{Himalayan rabbits} and \blank{arctic foxes} are sensitive to temperature, and change colour at different temperatures. This is caused by temperature sensitive \blank{tyrosine}.}

\flashcard{The environment affects the phenotype; a \blank{worse diet} can make a human twin grow to be smaller, and flowers have \blank{different colours} based on the soil \blank{pH}.}

\flashcard{Most \blank{phenotypes} are due to several genes and the environment (e.g. \blank{skin colour}, \blank{height}, \blank{weight}).}

\flashcard{A greater similarity between \blank{identical twins} for a particular \blank{trait} compared to \blank{fraternal twins} provides evidence that \blank{genetic} factors play a role.}

\flashcard{\blank{Identical} twins share all their genes and their home environment. \blank{Fraternal} twins share \blank{half} their genes and a home environment.}

\card{Define a mutation}{A \textbf{permanent} alteration in the DNA sequence
passed on into daughter cells (and sometimes gametes).}

\flashcard{The size of mutations ranges from \blank{a single base pair} (\blank{single nucleotide polymorphism} - SNP) to \blank{large segments of a chromosome} (\blank{chromosome rearrangement})}

\flashcard{SNP mutations are \blank{micro-mutations}, chromosome rearrangements
are \blank{macro-mutations}}

\card{Define a hereditary mutation.}{A mutation inherited from a parent gamete and present throughout a person's life and in every cell in their body. This can be passed on to progeny through meiosis.}

\card{Define an acquired (somatic) mutation.}{When a mutation occurs at some point in a person's life, and is present only in the cell that it occurred and it's daughter cells (through mitosis).}

\card{Environmental factors that cause mutations include...}{Mutagens; chemicals, radiation etc that causes breaks between DNA bases. Biological factors such as viruses that can integrate into the genome and cause disturbances in the DNA.}

\card{Intrinsic factors causing mutations include...}{Errors during DNA replication (before mitosis) and repair. Errors during meiosis (e.g. an error in chromosome separation).}

\flashcard{Macro mutations occur during \blank{meiosis} or in \blank{late stage cancers}}

\card{Mutations during meiosis include...}{Trisomy (when a sperm has an extra chromosome) or monosomy (when a sperm has one too few chromosomes).}

% TODO: Get images for these (slide 5, lecture 2)

\card{Single chromosome macro-mutations include...}{Within one chromosome; deletion, duplication and inversion of regions of the chromosome. Within two chromosomes, part of one can go into another (insertion), parts of chromosomes can swap places (translocation).}

\flashcard{Examples of diseases caused by macro-mutations include \blank{down syndrome}, \blank{klinefelter syndrome} and \blank{Cri du chat}.}

\card{What are the three types of substitution micro-mutations and what are they caused by?}{Caused by single base substitutions (SNP), and they are silent, nonsense (STOP) and mis-sense.}

\card{How does a nonsense mutation occur?}{When a SNP (single base substitution) converts a triplet from coding a protein to coding a STOP signal.}

\card{What is a silent mutation?}{When the protein coded for by a triplet is not changed by an SNP.}

\card{What is a mis-sense mutation?}{When a SNP mutation changes the protein coded for by a triplet.}

\flashcard{\blank{Insertions and deletions} can cause great disturbances to a protein through \blank{frameshift mutations} unless the number of bases \blank{is divisible by three}, so there is no \blank{frameshift}}.

\flashcard{There are \blank{900} bad (but \blank{recessive}) alleles for cystic fibrosis. The normal gene \blank{produces enough protein to compensate}. Patient must be \blank{homozygous} for one bad allele, or \blank{heterozygous} for two.}

\flashcard{\blank{Trinucleotide repeated expansions} are when a person has many repeats of a base pair triplet. \blank{The number of repeats} dictates the likelihood of a person getting certain diseases (more is worse for the patient).}

\flashcard{Sometimes a SNP in a region far away from a gene can cause problems. In the case of lactose intolerance, a pair 13910 bases before the relevant gene is substituted (from T to C), meaning a protein cannot bind. This is recessive, since just a bit of lactase does the job.}

\flashcard{The Human Genome project took \blank{13 years} to sequence \blank{3 billion} base pairs. DNA from \blank{5 anonymous} individuals of \blank{varying ethnicity} was taken.}

\flashcard{It was discovered that humans only have \blank{20,500} genes, but it was thought that humans should have around \blank{100,000}. This was because flies have \blank{13,000} and humans are more complicated!}

\flashcard{Humans share \blank{sixty percent} of their genes with flies, and only \blank{two percent} of the human DNA codes for genes.}

\card{Why can humans get by with so few genes?}{Alternative splicing; the same gene can produce different proteins when it is shaped differently (isoforms). This means that we can make 100k proteins with 23k genes.}

\flashcard{Cells have the \blank{same genome}, but do not express the \blank{same genes and isoforms}. Where these \blank{proteins} are expressed determines the type of cell formed.}

\flashcard{Humans genomes differ by about \blank{0.01 percent}, which is about \blank{3 million} base pairs which are mostly \blank{SNP's}}

\flashcard{The frequency of SNP's is one in every \blank{300} base pairs. Most are \blank{outside genes} and have \blank{no effect on the phenotype}.}

\card{SNP's outside of genes are useful because...}{They act as landmarks for us as scientists!}

\card{GWAS stands for...}{Genome wide association studies}

\flashcard{Most diseases result from \blank{polygenic and environmental interactions}, patients with \blank{particular groups of landmark SNP's} have been found to be more at risk of developing some diseases.}

\flashcard{GWAS aim to identify the common SNP's associated with \blank{complex diseases and traits} by testing at least \blank{hundreds of thousands} of SNP's in large population samples.}

\card{Where are the samples for GWAS taken from}{Both patients who have the disease and people who do not (the control).}

\flashcard{When particular landmark SNP's are seen in greater diseased patients compared to controls, we say that the SNP's are \blank{associated} with the disease.}

\card{If a patient has SNP's associated with a disease, what does it mean?}{The patient as a higher risk of the disease (very rarely, there could be a 100 percent association).}

\flashcard{Some people will be affected more by \blank{their environment} if they have SNP's associated to a disease in their genome (e.g. are far more likely to get a disease if they smoke).}

\card{What is pharmacogenomics?}{How do patients genomes affect their response to a treatment?}

\flashcard{In 2005, \blank{less than 50} SNP's were known to be associated with diseases, in 2008, it was \blank{over 500} and now it's over \blank{14,000}.}

\card{What was the aim of the 1000 genomes project?}{To establish the most detailed catalogue of human genetic variations.}

\flashcard{On average, each person carries \blank{250-300} loss of function variants in annotated genes, and \blank{50-100} previously implicated in inherited disorders.}

\card{How many new disease causing mutations were identified in the 1000 genomes project?}{671}

\flashcard{In \blank{2014} the 100,000 genomes project was started by \blank{the NHS}. It was split between helping \blank{cancer patients} and \blank{patients with rare diseases}.}

\flashcard{The 100,000 genomes project sampled \blank{75,000} people including \blank{40,000} serious illness patients. \blank{50,000} cancer patient genomes (one cancer and one normal per patient), and \blank{50,000} rare disease genomes (three per patient; \blank{one patient genome and two blood relatives)})}

\flashcard{\blank{23andMe} and \blank{Illumina} both let you get your genome sequenced. \blank{23andMe} does not offer much advice or counselling, but \blank{illumina} does, and is therefore more expensive.}

\flashcard{Immlumina tests healthy adults interested in learning about their risk for \blank{a set of adult-onset conditions}, assessing their \blank{carrier} status and understanding their response to certain \blank{drugs}.}

% Lecture 3

\card{How many different types of cell are there in humans?}{220 cell types.}

\card{What is the first cell created by the fusion of the egg and sperm?}{The zygote.}

\card{What are the initial cells formed from the zygote called?}{Blastomeres}

\card{After there are more than 8 blastomeres, what is there?}{A blastocyst a ball of cells.}

\card{What is the trophoblast?}{The embryo after it was a blastocyst (5 days). Separate from the inner cell mass}

\card{Where does the embryo form from?}{The inner cell mass, not the trophoblast.}

\flashcard{When the \blank{inner cell mass} is dividing, the cells become smaller since they are partitioning the \blank{zygote} cytoplasm via mitosis.}

\card{What lets the embryo attach to the wall of the uterus?}{The trophoblast}

\flashcard{\blank{uterine implantation} is driven by the \blank{trophoblast}. The \blank{Inner cell mass} expands and changes shape and location, but is still \blank{only one type of cell}.}

\flashcard{Once attached to the uterus wall, the inner cell mass sets the \blank{axis of the body}. The \blank{primitive streak} is the \blank{anterior posterior (head to tail)} axis.) The body is symmetrical along this.}

\flashcard{After setting the axis, \blank{gastrulation} takes place. This is where cells migrate, along the bottom, endoderm form, \blank{mesoderm} in the middle \blank{and ectoderm} at the top. \blank{ectoderm} will be the skin and nerves, \blank{mesoderm} forms \blank{muscles, blood, skeleton, heart etc} and the \blank{endoderm} forms the \blank{digestive system, lungs etc}}

\card{What is a highly coordinated cell movement?}{Gastrulation}

\card{What structures become the vertebrae?}{Somites; they emit signals telling what organs to form where.}


\card{What do somites eventually form into?}{Muscles, vertebral column and dermis of the skin. They are landmarks for organ formation during development.}

\card{Growing organs is called...}{Organogenesis}

\card{By saying organogenesis is progressive, we mean}{That the organs grow in stages, e.g. there is a little growth for the arm first, then it gets longer, then it gets digits etc.}

\card{What is used as a reference for growing specialised cells in an embryo?}{The head to tail framework.}

\card{What is a differentiated cell?}{One where the shape, structure and function is well defined.}

\card{The gurdon experiment was done on...}{Frogs}

\card{The gurdon experiment involves...}{Taking egg cells, removing the nuclei and inserting nuclei from either a small embryo or a developed intestine cell. The former usually develop into tadpoles, but the latter mostly stop developing before the tadpole stage.}

\card{Cells developmental potential (potency) changes how as it gets more specialised?}{It decreases.}

\card{What is involved in a grafting experiment?}{Cells from an early gastrula (early embryo) that would form an eye are taken and transplanted into an host embryo (oldest), as well as ones from an neurala (older embryo than gastrula). The ones from the younger embryo develop into anything depending where they are implanted, the ones from the older embryo develop into eyes.}

\flashcard{The fate of a cell \blank{can be locked} before differentiation. They can sometimes \blank{not adapt to} a new situation, up to \blank{4 generations before}.}

\flashcard{
  \small
  \begin{tabularx}{0.5\textwidth}{c|X|X|X}
    \textbf{Source} & \textbf{Potential} & \textbf{Type of cell} & \textbf{Can develop into}\\  \hline
    Zygote & \blank{Totipotent} & - & Whole organism.\\ \hline
    \blank{Blasocyst} & \blank{Pluripotent} and self-renewing & Embryonic stem cell & Any cell type\\ \hline
    Adult & Multipotent, \blank{self-renewing} & multipotent \blank{stem cells} & Some cell types\\ \hline
    Organ & Limited potential and renewal & \blank{Progenitor} & Choice of between \blank{2-6} types\\  \hline
    - & Limited division & committed progentor & 1 type, locked fate.\\ \hline
    - & No division & Differentiated & No division.
  \end{tabularx}
}

\card{Once a cell is differentiated...}{It has a clear cut identity and expresses specific proteins for morphology and function.}

\flashcard{Cells have the same genes, but it's how they express their genes that makes them different.}

\flashcard{At any given time, each cell expresses around \blank{20 percent} of it's genes}

\flashcard{About \blank{ten percent} of the \blank{20 percent} active genes are developmental genes.}

\card{Developmental genes control:}{Proteins that regulate genes expression (turn genes on and off), proteins involved in cell communication or signalling (tell other cells what genes to turn on and off).}

\flashcard{One small difference in gene expression can \blank{create a cascade of changes downstream}.}

\flashcard{Proteins inside the egg are \blank{not uniformly distributed}.}

\flashcard{After two \blank{cleavage divisions} of the zygote (egg to two cells, to four), the \blank{same maternal proteins} are in the cytoplasm. After division two, the cells have different maternal proteins after division, so they have different gene expressions and more differences occur after each cell division onwards.}

% TODO: How a beta insulin cell is made

\flashcard{A differentiated cell can give rise to a new organism (\blank{totipotent}), which means genes are \blank{not lost} as a cell specialises.}

\flashcard{\blank{Epigenetics} are proteins that bind to the DNA and \blank{retrieve totipotency}. They change how the DNA is shaped so that different parts can be accessed.}

\flashcard{
  The embryo starts with a zygote (\blank{totipotent}.\\
  It becomes a \blank{blastocyst} with a \blank{trophoblast} and \blank{ICM} (\blank{plurpiotent})\\
  Before full differentiation, cells \blank{become locked in their fate} and are \blank{determined}\\
  At the gene level, cells become different by \blank{expressing different developmental genes}\\
  The initial differences come from the  \blank{maternal developmental proteins} being unevenly distributed in the \blank{egg cytoplasm}. As \blank{blastomeres} form from \blank{cleavage divisions}, they end up not having the same \blank{developmental proteins}.
}

% Lecture 4

\flashcard{\blank{Totipotent} stem cells can become any cell.}

\flashcard{\blank{totipotent stem cells} have the minimum level of specialisation, \blank{differentiated cells} have the maximum level of specialisation.}

\flashcard{\blank{Committed progenitor} cells are not stem cells, but \blank{progenitor} cells are.}

\flashcard{\blank{8} cell stage is the limit for totipotency in humans.}

\flashcard{Stem cells in the ICM are \blank{pluripotent}.}

\flashcard{Adult stem cells are \blank{multipotent}.}

\card{Adult stem cells are found in...}{Brain, Skin, Bone Marrow, Skeletal muscle, Intestines (any cell that needs regrowth).}

\flashcard{Embryonic stem cells are \blank{immortal} and \blank{pluripotent}.}

\flashcard{In order to control ESL's in vitro, we can \blank{change the chemical composition of the} culture  medium, or \blank{insert specific genes into cells}.}

\flashcard{Adult stem cells are \blank{harder} to grow in the lab than \blank{ESC's} but do show \blank{some plasticity}.}

\card{Describe the plasticity of ASC's}{Most adult stem cells can trans-differentiate in the lab, but this is a low efficiency process.}

\card{What are the most apparently plastic cells?}{Mesenchymal stem cells (mesoderm), which can transform into liver cells (endoderm) and brain cells (ectoderm)}

% TODO: Tabularize
\card{Why are UC-MSC's better than BM-MSC's?}{
  \begin{tabular}{rl}
    -&Less immunogenic\\
    -&longer telomeres\\
    -&less DNA damage\\
    -&non-invasive to harvest\\
    -&same plasticity as BM-MSC's.
  \end{tabular}
}

\card{How many proteins are usually considered for immuno-compatibility?}{5}

\card{What is GVHD?}{Graft Vs Host Disease, where the immune cells in the transplant attack the host.}

\card{Why are neonatal (UC cells) less immunogenic?}{Embryos and foetuses have to evade the mother's immune system, so there are less surface markers on cells. Also, newly born babies have no/little immune system so there is less chance of GVHD.}

\flashcard{Neonatal cells have longer \blank{telomeres} (which \blank{indicate the age of the cell}), since they get shorter \blank{at each cell division} since they do not get replicated, and neonatal cells have not divided many times.}

\card{In ESC's what enzyme is expressed that stops a telomeres from getting shorter?}{Telomerase}

\card{When is telomerase turned off?}{Before the baby is born}

\card{What enzyme do most cancer cells produce and why?}{Telomerase so that the cells are immortal and divide indefinitely.}

\card{What is a bank of ESC lines?}{A bank of embryonic stem cells, where each `line' of cells is derived from a single embryo.}

\card{What are the three sources of human stem cells?}{Embryonic SC's, Neonatal SC's, adult SC's (bone marrow, fat tissue (liposuction), skin).}

\card{How could we make a stem cell with only some skin cells?}{Make it into an induced pluripotent stem cell in the lab.}

\card{What are the currently approved stem cell based therapies?}{Skin grafts, Hematopoietic SC transplant from adult bone marrow or neonatal cells.}

\card{How does a bone marrow transplant to cure leukaemia work?}{1. Get a matching donor 2. Replicate stem cells ex vitro 3. Destroy bone marrow in patient using irradiation and chemotherapy 4. transplant stem cells into patient.}

\card{Give an example of tissue engineering}{Remove cells from lungs, hips and nose, remove a donor trachea (from cadaver) and remove all cells, grow cells around trachea and transplant in patient.}

\card{What is ex-vivo and in-situ cartilage engineering}{Growing new cartilage outside the body and in the body respectively (using MSC's to stimulate growth).}

\card{List advantages of MSC's}{
  \begin{tabular}{rl}
    -&Easy to isolate\\
    -&Plastic (not literally!) in the lab\\
    -&Can be frozen and thawed\\
    -&Possess potent immuno-suppression and anti-inflammation effects\\
    -&Capable of homing (going to site of injury)\\
    -&Stimulate regeneration
  \end{tabular}
}

\flashcard{MSC's might be good for \blank{cotransplants} e.g. with HMC's since they help other stem cells to graft}

\flashcard{Clinical trials take \blank{a long time}, and \blank{less than 10} therapies are in phase 3 for stem cell treatments. Foreign clinics advertise MSC treatments, but none have published data from clinical trials.}

\flashcard{Most trials for stem cell therapies are carried out with MSC's (\blank{70 percent}), HSC's count for \blank{20 percent}. ESC's are around \blank{2 percent} and are being tested with \blank{eyes} since they are \blank{immuno-privileged}.}

\flashcard{SC's can be used for \blank{replacing cells (e.g. transplants)}, \blank{repairing cells (e.g. genetically modify SC's outside the body and re-implant)} and \blank{protecting via MSC immunosupression}.}

\card{For repairing and replacing cells, what type of cell should we use?}{The patients own cells (autologous transplants). This requires adult stem cells that are reasonably plastic though, and its hard to isolate ASC's in the lab. Otherwise, use donor SC's with low immunogenicity.}

\card{What is an induced pluripotent stem cell?}{When you reprogram a normal (e.g. skin) cell by inserting genes (via viruses or otherwise). Only 3-4 gene insertions required.}

\card{How to do Parkinson's in a dish?}{
  \begin{tabular}{rl}
    -&Collect skin cells\\
    -&Re-program them into stem cells\\
    -&Grow brain cells from them (induce brain\\
     &cell differentiation)\\
    -&Stress out the brain cells with toxins\\
    -&Observe Parkinson's-like features
  \end{tabular}
}

% Personalised medicine

\flashcard{The traditional approach to medicine is \blank{one size fits all}.}

\flashcard{The traditional approach to medicine does not take into account \blank{individual differences between patients}, which is successful for some, but not all patients.}

\card{What is stratified medicine?}{Targeting different types of specific diseases made up of lots of different genes e.g. maturity onset diabetes}

% TODO: Insert from slide 1 on personalised medicine...
\flashcard{Personalised medicine (aka \blank{precision medicine}) takes into account }

\card{Examples of historical personalised medicine include...}{Inheritance of alkaptunoria, blood transfusions using blood capability testing, genetic basis of selective toxicity of an antimalarial drug.}

\flashcard{When the human genome project started, \blank{4} drugs had pharmacogenetic information. After it ended, \blank{46} drugs had this information and ten years later, there \blank{104} drugs. Now the \blank{genome}, \blank{proteome}, \blank{metabolome} and \blank{epigenome} are examined.}

\flashcard{Genetic changes of interest include \blank{SNP's}, \blank{base insertions}, \blank{copy-number variations} and \blank{variable number tandem repeats}. These all change how much of the proteins coded for by an affected gene is produced.}

%TODO: Make not run over
\card{What are the advantages of personalised medicine (6 things)?}{
\begin{tabular}{rl}
-&Shift reaction to prevention\\
-&Predict susceptibility of developing a disease\\
-&Improve dosing of drugs (increase efficiency, reduce side effects)\\
-&Reduce cost, time and attrition rate in drug development\\
-&Decrease adverse affects of drugs, increase diagnostic and detection power for disease
\end{tabular}}

\card{What genes increase your risk of breast and ovarian cancer and how much by?}{BRAC1, BRAC2; 85 percent higher lifetime chance of breast cancer and 60 percent chance of ovarian cancer.}

\flashcard{There are over \blank{15000} predictive tests looking at \blank{2800} genes. They can \blank{save the cost} of treating patients.}

\flashcard{Even if a predictive test for a gene doesn't have an associated drug to lower risk, you can \blank{change environmental factors (e.g. eat better, stop smoking etc)}. Sergey Brin does this for Alzheimer's!}

\flashcard{It's easy to take biopsy of cancer tumours (because they're by definition, not needed), so they can have their genome sequenced to see what genes the cancers have.}

\flashcard{There are drugs (Ivacaftor) that target the \blank{gene underlying cause} of diseases rather than just treating symptoms.}

\card{What does metastatic cancer mean?}{When the cancer has moved from the original site to other areas of the body.}

\flashcard{Enzymes metabolise drugs, and \blank{one family of enzymes} metabolise over \blank{90} percent of drugs. There are \blank{thousands of mutations} in genes that code for these enzymes. Some people metabolise fast (and are at risk of \blank{overdose toxicity}), or even ultra-fast metabolisers (meaning the drugs \blank{are broken down before they have an effect}).}

% TODO: Flashcardify
\flashcard{After a stent has been put into an artery, the body recognises it as foreign and blood will clot around it. A drug is given to stop clotting, but one enzyme (CYP 2C19) converts the drug from inactive to active. Variations in this enzyme mean not as much is converted, meaning the blood can clot possibly causing a heart attack or stroke.}

\card{What are some problems with personalised medicine?}{Ethics, multiple gene variations per disease, quantity of data}

\card{What are the ethical problems with personalised medicine (5 things)?}{Who sees the data? How will it be stored? How will it be used? Could it be used against us? What legal protection do we have?}

\flashcard{\blank{Driver and passenger} mutations can be involve with genes. Drugs need to target driver mutations in order to be effective.}

% Lecture 2 of personalised medicine

\card{Define biomarker}{A naturally occurring molecule, gene or characteristic
by which a particular pathological or physiological process, disease etc can be identified.}

\card{Why are biomarkers helpful?}{Because they help with prediction, diagnosis, progression, regression, or the outcome of treatment of a disease.}

\flashcard{\blank{multiple biomarkers} can be used to build up a signature, telling us how multiple \blank{genes/proteins} etc contribute towards a disease}

\flashcard{\blank{GWAS} and \blank{microarrays} are used to identify genes involved in diseases.}

\flashcard{Given a patient, we can use \blank{DNA sequencing}, \blank{microarrays} and \blank{immunohistochemistry} (detecting antigens on the surface of cells) to determine their biomarkers. These can be gotten from \blank{normal or diseased tissue}, \blank{blood}, \blank{saliva}, \blank{sweat} etc. Anywhere where we \blank{can find protein or DNA} in the body.}

\card{How do DNA chips work?}{First, sample DNA is taken, then it is amplified using PCR. It is then placed on a DNA chip with many probes, where it will bind to probes that it is complementary to. The chip is washed to remove the non-bound DNA, then scanned, where the bound probes will be visible.}

\card{What is the name for a test that goes with a drug?}{A companion diagnostic (CDx).}

\flashcard{Oncotype Dx identifies \blank{16} genes associated with \blank{breast cancer} and \blank{5} housekeeping genes (used as a control). These are used to give a score of 1-100 giving the likely reoccurrence of \blank{the tumor} within the next \blank{ten} years. It also predicts the response to \blank{chemotherapy}. This costs \blank{\$4175}.}

\flashcard{\blank{MammaPrint} determines how aggressive a \blank{breast cancer} tumor is (i.e. whether there is a high or low risk of \blank{metastasis}). It measures the mRNA of \blank{1900} genes. A biopsy is taken and determined to make sure that \blank{thirty percent} or more cells are cancerous, then the tissue is used for a \blank{microarray}.}

% TODO: Are these slides (lecture 2 of personalised medicine) case studies?
% Can the content be on the exam?

% Biodiversity 1

\card{What is ecosystem services?}{Involves putting a value on a service that protects biodiversity. E.g. instead of building a dam, work out how much preseving a forest can help water retention (protecting water resources).}

%TODO: Slide 1 lecture 1 of biodiversity
\card{Give examples of ecosystem services.}{FILL IN}

\card{Biological resources include...}{Food, medinal resources and pharmaceutical drugs (e.g. stuff in rainforests), wood, ornamental plants, breeding stocks, gene diversity.}

\card{What are the social benefits to biodiversity?}{Research, recreation and tourism, culture.}

\card{Define biodiversity}{The variety of life at all levels; gene level, population level, species level, ecosystem level. Also, the interactions between these living things.}

\card{What are the three main levels of biodiversity.}{Genetic diversity, species diversity, ecosystem diversity.}

\card{Define genetic diversity}{The combination of different genes found within a population of a single species, and the pattern of variation found within different populations of the same species.}

\card{What does a low genetic diversity mean?}{The key factor driving the extinction vortex is the loss of genetic variation since variation is necessary for evolutionary responses to environmental change.}

\flashcard{A small population is prone to positive feedback loops that draw it down an \blank{extinction vortex}.}

\card{Define genetic drift}{When a population has little genetic variation, any disease that all individuals are suseptible to could kill the whole population.}

\flashcard{Extinction vortex: small population means interbreeding and genetic drift, so there is a loss of genetic diversity, meaning that there is a reduction in individual fitness and population adaptability so there is lower reproduction and a higher mortality.}

\card{What (is the biggest thing that) makes species susceptible to extinction?}{Small population size; i.e. rare species are most at risk.}

\flashcard{Cheetah has a \blank{low genetic variation} because it had a \blank{genetic bottleneck} near the last ice age (\blank{only a few individuals survived}) and they had an isolated populations in North Africa and Asia are \blank{still genetically similar}.}

%TODO: Case study fill in (greater prarie chicken):
\flashcard{Greater Prarie Chicken}

\card{Define species diversity}{The variety and abundance of different types of organisms which inhabit an area.}

\card{Define species richness}{The number of different species in a particular area.}

\card{Define species evenness}{The relative abundance with which each species is represented in an area (e.g. lots more grey than red squirels).}

\card{Define ecosystem diversity}{Encompasses the variety of habitats that occur in a ragion, or the mosaic of patches found within a landscape}

\flashcard{An ecosystem can \blank{cover a large area} such as a whole forest or a \blank{small area} such as a pond.}

\card{Give four causes of biodiversity loss}{Habitat loss, intriduced species, overexploitation, pollution.}

\flashcard{Most threatened species are imperiled \blank{for more than one reason}.}

\card{Give the three types of (endangered) species}{Rare, dominant and keystone}

\card{What is a rare species?}{A species that has a small population.}

\card{What is a dominant species?}{A species that supports many other species. Saving these often helps many others.}

\card{What is a keystone species?}{A keystone species is a species whos very presence contributes to a diversity of life and whos extinction would lead to the extinction of many otheres. They help support the whole ecosystem.}

\flashcard{Conserving biodiversity aims to look for \blank{hotspots} by mapping biodiversity by region (richness, levels of threat) and by country. These areas are ranked and preserved.}

\flashcard{Hotspots have three aspects; \blank{richness}, \blank{threatened species} and \blank{endemic (unique to a location) species}. These rarely overlap.}

\flashcard{There are \blank{34} hotspots in the world containing \blank{seventy five percent} of the worlds threatened vertebrates and only cover \blank{two point three percent} of the earth's surface.}

%TODO: Costa rica case study
\flashcard{FILL IN}

%TODO: Costa rica frog diversity
\flashcard{FILL IN}



\end{document} 
